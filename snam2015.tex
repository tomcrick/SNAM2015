\RequirePackage{fix-cm}
%
%\documentclass{svjour3}                     % onecolumn (standard format)
%\documentclass[smallcondensed]{svjour3}     % onecolumn (ditto)
%\documentclass[smallextended]{svjour3}       % onecolumn (second format)
\documentclass[twocolumn]{svjour3}          % twocolumn
%
%\smartqed  % flush right qed marks, e.g. at end of proof
%
\usepackage{graphicx}
%
% \usepackage{mathptmx}      % use Times fonts if available on your TeX system
%
% insert here the call for the packages your document requires
%\usepackage{latexsym}
% correct typesetting of URLs
\usepackage{url}
% correct typesetting of metric units
\usepackage[textstyle,squaren]{SIunits}
% maths
%\usepackage{amsthm}
%\theoremstyle{definition}
%\newtheorem{definition}{Definition}
% bib
\usepackage[semicolon,sort&compress,nonamebreak]{natbib}
% hyperlinks for references
\usepackage[plainpages=false,breaklinks]{hyperref}


% please place your own definitions here and don't use \def but
% \newcommand{}{}

% Insert the name of "your journal" with
\journalname{Social Network Analysis and Mining}

\begin{document}

\title{Measuring UK Crime Gangs: A Social Network Problem
%\thanks{Grants or other notes
%about the article that should go on the front page should be
%placed here. General acknowledgments should be placed at the end of the article.}
}
%\subtitle{Do you have a subtitle?\\ If so, write it here}

%\titlerunning{Short form of title}        % if too long for running head

\author{Giles Oatley \and Tom Crick}

%\authorrunning{Short form of author list} % if too long for running head

\institute{
  Giles Oatley \and Tom Crick
  \at Department of Computing \& Information Systems, 
  Cardiff Metropolitan University, Cardiff CF5 2YB, UK\\
  \email{\{goatley,tcrick\}@cardiffmet.ac.uk}
}

%cover letter
% Dear Editor-in-Chief,

% Please find enclosed our manuscript, "Measuring UK Crime Gangs: A
% Social Network Problem" by Oatley and Crick, which we would like to submit for
% publication as an original article in Social Network Analysis and
% Mining.

% To our knowledge, this is the first report showing preliminary
% outcomes from a study to tackle the problem of
% gang-related crime in the UK during the 2000s; we believe our
% findings would appeal to the readership of SNAM.

% Please address all correspondence to me at tcrick@cardiffmet.ac.uk; we
% confirm that this manuscript has not been published elsewhere and is
% not under consideration by another journal. All authors have approved
% the manuscript and agree with its submission to SNAM.

% We look forward to hearing from you at your earliest convenience.

% With kind regards,

% Dr Tom Crick
% tcrick@cardiffmet.ac.uk

\date{Received: date / Accepted: date}
% The correct dates will be entered by the editor

\maketitle

\begin{abstract}
This paper describes the output of a study to tackle the problem of
gang-related crime in the UK; we present the intelligence and
routinely gathered data available to a UK regional police force, and
describe an initial social network analysis of gangs in the Greater
Manchester area of the UK between 2000-2006.

By applying social network analysis techniques, we attempt to detect
the birth of two new gangs based on local features (modularity,
cliques) and global features (clustering coefficient). Thus for the
future, identifying the changes in these can help us identify the
possible birth of new gangs (sub-networks) in the social system.

Furthermore, we study the dynamics of these networks globally and
locally, and have identified the global characteristics that tell us
that they are not random graphs -- they are small world graphs --
implying that the formation of gangs is not a random event. However,
we are not yet able to conclude anything significant about scale-free
characteristics due to insufficient sample size.

A final analysis looks at gang roles, and develops further insight
into the nature of the different link types, along the lines of
Klerk’s `third generation' analysis.

\keywords{Social network analysis \and Scale-free networks \and
Small-world networks \and Gangs \and Gun crime \and Social distance}
% \PACS{PACS code1 \and PACS code2 \and more}
% \subclass{MSC code1 \and MSC code2 \and more}
\end{abstract}


\section{Introduction}\label{sec:introduction}

There have been numerous studies of criminal networks and gangs; as
highlighted in \citet{hughes:2005}, the popularity of qualitative
studies of gang-related issues soared during the 1980s and 1990s,
following renewed media and public interest, statistical advances, and
increased government funding. Qualitative studies have taken three
major forms: (a) surveys of law enforcement officials (and at times
other agency personnel) regarding gangs in their jurisdictions and
actions taken to control them, (b) analyses of data compiled by law
enforcement agencies and/or court officials, and (c) self-reports of
samples of youth and/or young adults.  There have been calls for
research evidence to be drawn into police practice, but development of
such an agenda has been hampered by a range of
factors~\citep{bullock+tilley:2009}. Research into youth gangs,
especially the age at which youths join gangs and the early
precursors, has been conducted in the USA and
Canada~\citep{hill-et-al:2001}, China~\citep{webb-et-al:2011} and Hong
Kong~\citep{lo:2011}.

However, the UK has been slow in carrying out research into gang crime
(excepting work done in North East London~\cite{pitts:2007} in 2007),
and especially into what actions work best at controlling it. In
Greater Manchester, a region in the north of the UK that has had a
significant gun crime problem related to gang activity, primarily due
to acute social deprivation in the
area~\citep{BBCNews2003,BBCNews2004,HalesLewisSilverstone2006}, recent
police initiatives have started to address this
problem~\citep{BBCNews2010}.

\begin{acknowledgements}
We acknowledge the UK's Engineering \& Physical Sciences Research
Council (EPSRC) Sandpit on Gun Crime (September 2005, Warwickshire,
UK), funded by the IDEAS Factory; and the assistance of Xcalibre, the Greater
Manchester Police's specialist gang crime task force.
\end{acknowledgements}

% BibTeX users
\bibliographystyle{spbasic}      % basic style, author-year citations
\bibliography{snam2015}   % name your BibTeX data base

\end{document}

